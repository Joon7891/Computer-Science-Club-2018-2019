\section{Annuity}
In order to save up for the holiday season, Jacob deposits $K$ dollars into an account that pays $r$ percent per month. This is known as an annuity—the sum of a series of regular payments made at \textbf{equal} time intervals into an account which pays interest. The investment amount for Jacob's $i$-th deposit can be expressed as $$A_i = K(1 + \frac{r}{100})^{N - i}$$ where $A_1$ is the first payment and $A_N$ is the last payment. \\

Help him determine the amount of money in his account (the annuity) after he has made $N$ deposits.

\inputformat
The first line of input will contain two space-separated integers $K$ and $N$ denoting the amount of the regular payment and the amount of regular payments respectively. The next line of input will contain a single integer $r$ denoting the interest rate of the account.

\outputformat
Output the value of Jacob's account after $N$ deposits rounded to two decimal places.

\addsample
{
    500 4 \\
    4
}
{
    2123.23
}