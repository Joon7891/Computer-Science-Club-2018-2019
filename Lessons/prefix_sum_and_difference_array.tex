\documentclass{beamer}
\usepackage[utf8]{inputenc}
\usepackage{listings}
\usepackage{tikz}
\usetikzlibrary{positioning,shapes,fit,arrows}
\graphicspath{ {./images/} }

\lstset
{
    language=python,
    breaklines=true,
    basicstyle=\tt\scriptsize,
    keywordstyle=\color{blue},
    identifierstyle=\color{magenta},
}

\usetheme{Antibes}

\usecolortheme{beaver}
\usefonttheme[onlymath]{serif}

% \usecolortheme{crane}
\setbeamertemplate{navigation symbols}{}%remove navigation symbols

\title[Prefix Sum and Difference Arrays---Computer Science Club]{\textbf{Prefix Sum and Difference Arrays}}
% \author{Shon Verch}
\institute{Stephen Lewis Secondary School \\[3ex] {\large Computer Science Club}}
\date{December 7, 2018}

\newcommand{\sectionFrame}[3]
{
    \section{#3}
    \begin{frame}
    \begin{block}{}
    \begin{center}
        \Huge{#1}\\[0.5ex]
        \large{#2}
    \end{center}
    \end{block}
    \end{frame}
}

\begin{document}

\begin{frame} 
\titlepage 
\end{frame} 

\section{Prefix Sum Array}

\begin{frame}{Problem}
\begin{itemize}
    \item You are given an array of $n$ number and $q$ queries.
    \item A query consists of indices $i$ and $j$, for which you are to output the sum of the subarray from index $i$ to $j$, inclusive.
    \item What is the most optimal solution to this problem?
\end{itemize}
\end{frame}

\begin{frame}{Example}
    \begin{itemize}
        \item $n = 5, q = 3$
        \item Array:
        \begin{itemize}
            \item $[-4, 2, 6, -1, 7]$
        \end{itemize}
        
        \item Queries:
        \begin{itemize}
            \item Query 1: $i = 2, j = 4$
            \begin{itemize}
                \item Subarray: $[6, -1, 7]$
                \item Sum: $6 - 1 + 7 = 12$
            \end{itemize}
            
            \item Query 2: $i = 0, j = 3$
            \begin{itemize}
                \item Subarray: $[-4, 2, 6, -1]$
                \item Sum: $-4 + 2 + 6 - 1 = 3$
            \end{itemize}
            
            \item Query 2: $i = 3, j = 4$
            \begin{itemize}
                \item Subarray: $[-1, 7]$
                \item Sum: $-1 + 7 = 6$
            \end{itemize}
        \end{itemize}
    \end{itemize}
\end{frame}


\begin{frame}[fragile]{Intuitive Solution}
    \begin{lstlisting}
        def get_subarray_sum(array, i, j):
            total = 0
            
            for n in range(i, j + 1):
                total += array[n]
            
            return total
    \end{lstlisting}
    
    \begin{itemize}
        \item Setup: $\mathcal{O}(1)$
        \item One Query: $\mathcal{O}(n)$
        \item All Queries: $\mathcal{O}(qn)$
    \end{itemize}
    
    There needs to be a better way!
\end{frame}

\begin{frame}{Brief Explanation of Prefix Sum Array}
    \begin{itemize}
        \item Let $S(i, j)$ represent the subarray sum from $i$ to $j$ inclusive. \\
        \item Let $a_n$ represent the $n$-th indexed element in the array
    \end{itemize}
    
    As such, 
    \begin{equation*}
        S(i, j) = a_{i} + a_{i + 1} + \dots + a_{j - 1} + a_{j}
    \end{equation*}
    
    
\end{frame}

\begin{frame}[fragile]{Prefix Sum Array Solution}
    \begin{lstlisting}
        def set_up_psa(array):
            psa = []
            
            for i in range(len(array)):
                if i == 0:
                    psa.append(array[0])
                else:
                    psa.append(psa[i - 1] + array[i])
        
            return psa
        
        def get_subarray_sum(psa, i, j):
            if i == 0:
                return psa[j] - psa[i - 1]
            else:
                return psa[j]
        
            return psa[j] - psa[i]
    \end{lstlisting}
    
    \begin{itemize}
        \item Setup: $\mathcal{O}(n)$
        \item One Query: $\mathcal{O}(1)$
        \item All Queries: $\mathcal{O}(q + n)$
    \end{itemize}
\end{frame}

\end{document}