\section*{J2 - A Chemistry Question}
According to the law of conservation of mass, when there is a chemical reaction, the mass of the reactants is the same as the mass of the products. Thus, a chemical equation is balanced when the number of atoms in the reactants is equal to the number of atoms in the products. For example, the following reaction $$N_2 + H_2 \rightarrow NH_3$$,
is not balanced because there is one less nitrogen and one more hydrogen on the product side. Conversely, the chemical equation $$2H_2+O_2 \rightarrow 2H_2O$$, is balanced because the number of oxygen and hydrogen atoms are the same on both sides of the equation. 

Jean would like to determine whether a chemical equation is balanced given the number of atoms, for each element, in the reactant and product where $R_i$ and $P_i$ represent the number of atoms for the $i$-th element respectively. An equation is balanced if and only if $R_i = P_i$ for all $1 \leq i \leq N$ where $N$ represents the number of elements.

\inputformat
The first line of input contains a single integer $N$ denoting the number of elements. The next $N$ lines of input each contain two space-separated integers $R_i$ and $P_i$ describing the number of atoms in the reactant and product for the $i$-th element respectively.

\constraints
$1 \leq N \leq 10^6$ \\
$1 \leq R_i, P_i \leq 1,000$

\outputformat
Output \texttt{``yes''} if the equation is balanced; otherwise, output \texttt{``no''}.

\addsample
{
    2 \\
    2 2 \\
    1 4
}
{
    no
}

\addsample
{
    3 \\
    5 5 \\
    9 9 \\
    7 7
}
{
    yes
}