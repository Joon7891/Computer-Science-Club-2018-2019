\section*{J4 - $-\frac{1}{12}$}
Summing up natural numbers is fun - that is until someone tells you that the sum of all natural numbers is $−\frac{1}{12}$. While this is interesting and mind-boggling, this does not apply to you because today you will be examining some finite quantity of natural numbers, not \textit{all} natural numbers.

As a mental and physical exercise, you decide to sum up all the natural numbers between $a$ and $b$ inclusive, by hand. However, this soon ends up being very time consuming and so your good friend Steven decides to write a program to determine this sum for you.

Excited at the ability to quickly sum up all the natural numbers in a given range, you decide to plug away various values of $a$ and $b$ into Steven's program before coming to an important realization - Steven can't solve problems equivalent to Junior CCC problems. He can solve hard problems efficiently using centroid decomposition, Kruskal's algorithm, segment trees, and a variety of different techniques, but somehow he can't create programs to solve simple problems efficiently.

Will you show Steven an efficient example?

\inputformat
The first and only line of input will consist of two integers $a$ and $b$.

\constraints
For all subtasks, $1 \leq a, b \leq 10^9$

\textbf{Subtask 1 [10\%]} \\
$|a - b| \leq 10^6$ \\

\textbf{Subtask 2 [20\%]} \\
$|a - b| \leq 10^9$

\outputformat
Output the sum of all natural numbers between $a$ and $b$, inclusive.

\addsample
{
    5 10
}
{
    45
}