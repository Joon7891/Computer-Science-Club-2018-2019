\section*{J3 - ICS4U1}
Teaching an ICS4U1 class is hard, especially when there's $24$ people in the class rather than a more manageable $10$. The students are loud, walk in and out of class as they please, and argue about whose making CCO, making it very difficult for any teacher to teach ICS4U1.

While there may not be a solution to remedy the loudness or the CCO arguments, there is a simple solution to students walking in and out of class - fences! A simple rectangular fence can be installed around the students to ensure that students can no longer allowed to walk out of class as they please.

Of course, there are a number of considerations when installing this fence, cost and student laziness. To minimize the cost of installing this fence, the fence should be as small as possible to minimize material and labour costs. In addition, ICS students are extremely lazy - meaning you can't ask them to move inwards to allow for a smaller fence. However, you can place a fence on top of them.

Determine the minimum cost to install the fence!

\inputformat
The first line of input contains two space-seperated integers $N$ and $C$, the number of students and the cost of installing 1 unit of fencing. The next $N$ lines of input contain two integers $x$ and $y$, indicating that a student is currently at the point $(x, y)$ in the classroom.

\constraints
$4 \leq N \leq 5\;000$ \\
$1 \leq C \leq 100$ \\
$-1\;000 \leq x, y \leq 1\;000$

\outputformat
The output consists of a single integer, the minimum cost of the fence.

\addsample
{
    4 2 \\
    1 2 \\
    2 3 \\
    3 1 \\
    4 4
}
{
    24
}