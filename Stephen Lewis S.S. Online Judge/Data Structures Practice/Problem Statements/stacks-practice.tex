\documentclass{problem-set}

\title{Data Structures Practice - Stacks}
\subheader{Stephen Lewis S.S. Computer Science Club}
\date{Janurary 1, 2018}
\author{Joon Song}

\begin{document}
\maketitle

Joon is practicing data structures. He comes across the following stacks problem:

You initially have a stack of $N$ ($1 \leq N \leq 1,000$) elements. There are $M$ ($1 \leq M \leq 5,000$) operations you need to perform on it.

Each operation is one of the following:
\begin{itemize}
    \item $I$ $x$ - insert an element of value $x$ onto the top of the stack
    \item $R$ - remove the element on top of the stack
    \item $T$ - output the element on top of the stack
    \item $P$ - remove and output the element on top of the stack
    \item $N$ - output the number of elements in the stack
\end{itemize}

It is guaranteed that all elements in the array will be between $1$ and $10^9$ (inclusive).

\inputformat
The first line of input will contain integers $N$ and $M$.

The second line of input will have $N$ integers. The first element will be at the bottom of the stack, whereas the last element will be at the top of the stack.

The next $M$ lines will have an operation in the format described above.

\outputformat
For each $T, P,$ or $N$ operation, output the answer on its own line.

\newpage

\addsample
{
    10 10 \\
    42 18468 6335 26501 19170 15725 11479 29359 26963 24465 \\
    I 28145 \\
    R \\
    T \\
    R \\
    R \\
    I 11942 \\
    T \\
    R \\
    R \\
    N 
}
{
    24465 \\
    11942 \\
    7
}


\end{document}