\documentclass{problem-set}
\newcommand{\RNum}[1]{\uppercase\expandafter{\romannumeral #1\relax}}

\title{Team Preliminary \RNum{1}}
\subheader{Stephen Lewis S.S. Computer Science Club}
\date{December 14, 2018}
\author{Joon Song & Shon Verch}

\begin{document}
\maketitle

\section{$\boldsymbol{A}$ to the $\boldsymbol{B}$}

In math class, Greg is learning about radical functions and their inverse. In order to sharpen his mathematical rigour, he has decided to exercise his computational skills by determining $A$ to the $B$; however, he quickly realizes that this problem is much too difficult. Thus, he needs your help to determine $A$ to the $B$. Rescue Greg from mathematical damnation!

\inputformat
The first line of input will contain an integer $t$ denoting the number of computations Greg needs to do. The next $t$ lines contain two space-separated integers $A$ and $B$.

\constraints
$1 \leq t \leq 100000$\\
$0 \leq A \leq 20$\\
$0 < B \leq 20$

\outputformat
For each computation, output the value of $A^B$.

\addsample{
2\\
2 4\\
-8 6
}{
16\\
262144
}

\newpage
\section{Holiday Shopping}
The holidays have finally arrived, meaning it's Boxing Day season! Shon has $m$ dollars in his bank account and proceeds to make $n$ purchases of values $a_1, a_2, \dots, a_n$. Determine the state of Shon's bank account (deficit, zero, or surplus), along with the amount of the balance after the shopping spree.

\inputformat
The first line of input will contain the integer $m$, denoting the balance of Shon's bank account. The second line of input will contain the integer $n$, the number of purchases Shon has made. The third line of input will contain $n$ integers $a_1$, $a_2$, $\ldots$, $a_n$ representing the cost of purchase $i$, separated by a single space.

\outputformat
The first line of input should contain his bank account state: \texttt{``Deficit''}, \texttt{``Even''}, or \texttt{``Surplus''}. The second line of input should contain the balance after the shopping spree.

\addsample{
400 \\
4 \\
150 200 90 35}
{
Deficit \\
-75
}

\newpage
\section{Present Delivery}
% prefix-sum array (santa delivers presents to n houses where h_i denotes the amount of presents delivered to house i. Determine the amount of presents delivered from house i to house j). 

\section{Every Mom's Nightmare}
% 0-1 knapsack (mom has k dollars and wants to maximize satisfaction)

\end{document}